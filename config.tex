\title{ElegantBook:优美的 \LaTeX{}笔记——数学基础篇}
\subtitle{Elegant\LaTeX{} 李群与李代数}

\author{mronebian}
\institute{Elegant\LaTeX{} Program}
\date{\today}
%\version{4.5}
%\bioinfo{邮箱}{}

\extrainfo{忽有春风化剑意,斩我二十少年气}

\setcounter{tocdepth}{3}

% \logo{logo-blue.png}
\cover{figure/acs4}

% 修改标题页的橙色带
\definecolor{customcolor}{RGB}{245, 250, 246}
\colorlet{coverlinecolor}{customcolor}
% Packages
%\usepackage{standalone}
\usepackage{subfiles}%多文件管理

\usepackage{amsmath}%数学符号
\usepackage{amssymb}
\usepackage{amsfonts}
\usepackage{mathtools}%引入右侧大括号使用

\usepackage{graphicx}   %用于插入和处理图形
\usepackage{tikz}
\usetikzlibrary{positioning}
% 绘图所需的TikZ子库

\usepackage{booktabs} % 引入 booktabs 宏包以创建三线表
\usepackage{tabularx}%控制表格长度


% 设置主字体为仿宋(FZSSK.TTF),加粗字体为黑体(FZHTK.TTF),斜体为楷体(FZKTK.TTF)
\setCJKmainfont[
	Path={E:/postgraduate/note/font_2/},
	BoldFont={FZHTK.TTF},
	ItalicFont={FZKTK.TTF},
	BoldItalicFont={FZHTK.TTF} % 将粗斜体指定为黑体
]{FZSSK.TTF}
% % 设置无衬线字体为楷体(FZKTK.TTF),加粗字体为黑体(FZHTK.TTF)
% \setCJKsansfont[Path={E:/postgraduate/note/font_2/}, BoldFont={FZHTK.TTF}]{FZKTK.TTF}
%
% % 设置等宽字体为仿宋(FZFSK.TTF),加粗字体为黑体(FZHTK.TTF)
% \setCJKmonofont[Path={E:/postgraduate/note/font_2/}, BoldFont={FZHTK.TTF}]{FZFSK.TTF}
%
% % 定义新的字体族 zhsong,为仿宋(FZSSK.TTF)
% \setCJKfamilyfont{zhsong}[Path={E:/postgraduate/note/font_2/}]{FZSSK.TTF}
%
% % 定义新的字体族 zhhei,为黑体(FZHTK.TTF)
% \setCJKfamilyfont{zhhei}[Path={E:/postgraduate/note/font_2/}]{FZHTK.TTF}
%
% % 定义新的字体族 zhkai,为楷体(FZKTK.TTF),加粗时用黑体(FZHTK.TTF)
% \setCJKfamilyfont{zhkai}[Path={E:/postgraduate/note/font_2/}, BoldFont={FZHTK.TTF}]{FZKTK.TTF}
%
% % 定义新的字体族 zhfs,为仿宋(FZFSK.TTF),加粗时用黑体(FZHTK.TTF)
% \setCJKfamilyfont{zhfs}[Path={E:/postgraduate/note/font_2/}, BoldFont={FZHTK.TTF}]{FZFSK.TTF}
%
% % 设置无衬线字体为楷体(FZKTK.TTF),加粗字体为黑体(FZHTK.TTF)
\setCJKsansfont[Path={mnt/e/postgraduate/note/font_2/}, BoldFont={FZHTK.TTF}]{FZKTK.TTF}

% 设置等宽字体为仿宋(FZFSK.TTF),加粗字体为黑体(FZHTK.TTF)
\setCJKmonofont[Path={mnt/e/postgraduate/note/font_2/}, BoldFont={FZHTK.TTF}]{FZFSK.TTF}

% 定义新的字体族 zhsong,为仿宋(FZSSK.TTF)
\setCJKfamilyfont{zhsong}[Path={mnt/e/postgraduate/note/font_2/}]{FZSSK.TTF}

% 定义新的字体族 zhhei,为黑体(FZHTK.TTF)
\setCJKfamilyfont{zhhei}[Path={mnt/e/postgraduate/note/font_2/}]{FZHTK.TTF}

% 定义新的字体族 zhkai,为楷体(FZKTK.TTF),加粗时用黑体(FZHTK.TTF)
\setCJKfamilyfont{zhkai}[Path={mnt/e/postgraduate/note/font_2/}, BoldFont={FZHTK.TTF}]{FZKTK.TTF}

% 定义新的字体族 zhfs,为仿宋(FZFSK.TTF),加粗时用黑体(FZHTK.TTF)
\setCJKfamilyfont{zhfs}[Path={mnt/e/postgraduate/note/font_2/}, BoldFont={FZHTK.TTF}]{FZFSK.TTF}


% 定义快捷命令 \songti 代表仿宋字体
\newcommand*{\songti}{\CJKfamily{zhsong}}

% 定义快捷命令 \heiti 代表黑体字体
\newcommand*{\heiti}{\CJKfamily{zhhei}}

% 定义快捷命令 \kaishu 代表楷体字体
\newcommand*{\kaishu}{\CJKfamily{zhkai}}

% 定义快捷命令 \fangsong 代表仿宋字体
\newcommand*{\fangsong}{\CJKfamily{zhfs}}

